\documentclass{article}
\usepackage{physics}
\usepackage{amsmath}
\usepackage[margin=1in]{geometry}
\usepackage{amssymb}
\title{QM Scattering}
\begin{document}
\maketitle
\section*{Scattering recap}
We have an incident plane wave in the form of $ \Psi_{inc}= e^{ik z}$, and the wave scatter after some local potential with $ \Psi_{scatt} = f( \theta  \phi) \frac{e^{ikr}}{r}$ where our scattering wave is spherically symmetric.

\section*{Partial wave expansion}
We exploited the symmetric and used spherical polar coordinaes where our solution is $ \phi$ indepedent. We have the following:

\begin{equation}
e^{ikz} = e^{ikr  \cos  \theta} = \sum_{i=0}^{ \infty} i^l (2l+1) j_l (kr) P_l  \cos  (\theta)
\end{equation}
Now we need to solve the Schrodinger equation for $V(r)$. \\ 
We showed last time that 
\begin{equation}
 \psi_k(r) = \sum_{l=0} \frac{u_l (r)}{r}  P_l ( \cos  \theta)
\end{equation} 
where $u(r)$ satisfy the equation 
\begin{equation}
  \left[ \frac{- hba^2}{2m}  \frac{d^2}{d r^2} + \frac{ \hbar^2 l (L+1)}{2 m r^2} + V(r) \right]u(r) = \frac{ \hbar^2 k^2}{2m}  u(r)
\end{equation}
but as $r \rightarrow  \infty$, equation $3$ becomes 
\begin{equation*}
u'' + k^2 u = 0
\end{equation*}
which gives the solution $u_l \sim_{r \rightarrow  \infty} A_l  \sin(kr - \frac{l  \pi}{2} +  \delta_l )$. Therefore,
\begin{equation}
  \psi_k (r) sim_{r \rightarrow  \infty} \sum_{l} \frac{A_l}{2 i} \left[ \frac{e^{i(kr - l \frac{ \pi}{2} +  \delta_l)}}{r} -\frac{e^{-i(kr- \frac{l  \pi}{2}  ) +  \delta_l}}{r}      \right] P_l  (\cos  \theta)
\end{equation}
Where the first wave term is the out going radial wave, and the incoming radial wave comes from the expansion for the incoming plane wave.
\\  
\\
This means that the only incoming piece should arise from $ \Psi_{inc} = e^{ikz} $, and we should look at $ \Psi_{inc}$ more carefully. Specifically, we look at $(1)$ at large $r$, and realize that the only $r$ dependency resides in $j_l (kr)$. Therefore, we can re-write $(1)$ as 
\begin{equation}
e^{ikz} \sim \sum_{l=0}^{ \infty} i^l (2l+1)  \sin( \frac{kr - l  \pi / 2}{kr}) P_l( \cos  \theta)
\end{equation}we re-write $i^l  \sin(\frac{kr - l  \pi /2}{kr} )$ as $e^{i l \frac{ \pi}{2} } (\frac{e^{i(kr - l  \pi/2)} -e^{-i( kr - l  \pi /2)}}{2i} ) = \frac{e^{ ikr} - e^{i(kr - l  \pi)}}{2 i} $ Therefore, we can re-write $(5)$ as 

\begin{equation}
e^{ikz} \sim \sum_{l=0}^{ \infty} \frac{2l + 1}{2ik} \left(\frac{e^{ikr}}{r} - \frac{e^{i(kr- l  \pi)}}{r} \right) P_l(  \cos  \theta)
\end{equation}
And the incoming spherical wave in $(4)$ must match with the incoming spherical wave in $(6)$, and the incoming plane wave is the only part of the wave that carry the incoming spherical wave component.
\\ 
Therefore, we can match it term by term between $(4)$ and $(6)$ for the incoming spherical wave. Therefore,
\begin{equation}
-\frac{A_l}{2i} e^{il  \pi/2} e^{-i   \delta_l} =-\frac{2 l + 1}{2 i k} e^{i l  \pi} \implies A_l = \frac{2l + 1}{k} e^{i l  \pi / 2} e^{i  \delta_l}
\end{equation}
substitute the result of $(7)$ into $(4)$ as $r \rightarrow  \infty$ and you get 
\begin{equation}
  \psi_k(r) \sim_{r \rightarrow  \infty} e^{ikz} + \sum_{l = 0}^{ \infty}(2l+1) \left[ \frac{e^{2i  \delta_l}-1}{2ik} \right]P_l( \cos  \theta) \ \frac{e^{ikr}}{r} 
\end{equation}
This equation will be derived in however
\\ 
where the sum part of equation $(8)$ is nothing but $f( \theta)$, and we can write it as:
\begin{equation}
f( \theta) = \frac{1}{k} \sum_{l = 0 }^{ \infty} (2l +1) e^{i  \delta_l}  \sin  (\delta_l)P_l  (\cos  \theta)
\end{equation}
We want to find $ \delta_l (k)$, and this will give us $f( \theta)$ via $(9)$, and we can calculate $\sigma ( \theta) = \abs{f ( \theta)} ^2$. 
\\ 
We can also calculate that $\sigma_{tot} = \int \ d\Omega \abs{f( \theta)} ^2$, where we can exploit the orthogonality of $P_l$. This gives $\sigma_{tot} = \frac{4  \pi }{k^2} \sum_{l = 0 }^{ \infty} (2 l + 1)  \sin^2 ( \delta_l) = \sum_{l = 0 }^{ \infty} \sigma_l$, where $\sigma_l \equiv \frac{4  \pi }{k^2} (2 l + 1)  \sin^2 ( \delta_l) \leq \frac{4  \pi }{k^2} (2 l + 1)$.

\subsection*{Example}
$V(r) =  \infty \ \text{for } r<r_0$, and $V(r) = 0$ elsewhere. \\ 
We have our radial wave equation 
\begin{equation*}
 \psi_k(r) = \sum_{l} R_l (r) P_l( \cos  \theta)
\end{equation*}
where 
\begin{equation*}
R_l(r) = A_l j_l (kr)+ B_l n_l (kr)
\end{equation*}
Normally, we only have $j_l$ in our $R_l$ because $n_l$ misbehaves near $r=0$. However, in our case, as the wave function must vanish at $r= r_0$. We actually \textbf{can} accept $n_l$ as a valid set of solution.
\\ 
Because $R_l (r_0) = 0$, we can get $\frac{B_l}{A_l} = - \frac{j_l (k r_0)}{n_l (k r_0)} $. Next time, we will use this to compute $ \delta_l$ and $\sigma_l$



















\end{document}
